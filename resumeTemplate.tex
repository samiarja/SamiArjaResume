%%%%%%%%%%%%%%%%%%%%%%%%%%%%%%%%%%%%%%%%%
% Medium Length Professional CV
% LaTeX Template
% Version 2.0 (8/5/13)
%
% This template has been downloaded from:
% http://www.LaTeXTemplates.com
%
% Original author:
% Trey Hunner (http://www.treyhunner.com/)
\usepackage[utf8]{inputenc}
\usepackage[english]{babel}
\usepackage{hyperref}
\usepackage{fontawesome}
\usepackage{xcolor}
\usepackage{hyperref}
% Important note:
% This template requires the resume.cls file to be in the same directory as the
% .tex file. The resume.cls file provides the resume style used for structuring the
% document.
%
%%%%%%%%%%%%%%%%%%%%%%%%%%%%%%%%%%%%%%%%%

%----------------------------------------------------------------------------------------
%	PACKAGES AND OTHER DOCUMENT CONFIGURATIONS 
%----------------------------------------------------------------------------------------
\renewcommand{\acvHeaderSocialSep}{\quad\textbar\quad}

\makeatletter
\patchcmd{\@sectioncolor}{\color}{\mdseries\color}{}{}
\makeatother
\documentclass{resume} % Use the custom resume.cls style 
\usepackage{fontawesome}
\usepackage{xcolor}
\usepackage[dvipsnames]{xcolor}
\usepackage{bibentry}
\usepackage{hyperref} 
\usepackage[left=0.4 in,top=0.3 in,right=0.4 in,bottom=0.3in]{geometry} % Document margins
\newcommand{\tab}[1]{\hspace{.2667\textwidth}\rlap{#}}
\newcommand{\itab}[1]{\hspace{0em}\rlap{#1}}

\name{Sami Arja}

%\address{123 Pleasant Lane \\ City, State 12345} % Your secondary address (optional) 
\address{\textbf{Software Engineer}\hspace{0.5cm}\textbf{Research Assistant}\hspace{0.5cm}\textbf{ML Developer}}
\address{\faMobilePhone \hspace{0.1cm} +61 424403034 \hspace{0.5cm} \faEnvelope \hspace{0.1cm} sami.arja@gmail.com \hspace{0.5cm} \faGithub \hspace{0.1cm} samiarja \hspace{0.5cm} \faLinkedin \hspace{0.1cm} samiarja} % Your phone number and email

\begin{document}  
\nobibliography{ref}
\bibliographystyle{unsrt}
%----------------------------------------------------------------------------------------
%	EDUCATION SECTION
%----------------------------------------------------------------------------------------

\begin{rSection}{Education}


{\bf Master of Philosophy in Neuromorphic Engineering (Scholarship)} \hfill {September 2019 - September 2021}
\\
\textbf{Thesis:} Neuromorphic Perception in Greenhouse Technology using 
\\
Event-based Vision Sensors
\\
International Center of Neuromorphic Systems (ICNS)
\\
The MARCS Institute
\\
 Western Sydney University, Werrington 

{\bf Bachelor of Engineering in Robotics and Mechatronics (Honours) \faGraduationCap} \hfill {February 2015 - June 2019}
\\
\textbf{Thesis:} Deep Convolutional Neural Network for Human Activity Recognition
\\ 
 Western Sydney University, Kingswood 
\\
Honour Class II Division I


{\bf Diploma in Engineering \faGraduationCap} \hfill {February 2014 - December 2015}
\\ 
 Western Sydney University, Nirimba 



\end{rSection} 

%----------------------------------------------------------------------------------------
%	TECHNICAL STRENGTHS SECTION
%----------------------------------------------------------------------------------------

\begin{rSection}{skills and INTERESTS}

\begin{tabular}{ @{} >{\bfseries}l @{\hspace{6ex}} l } 

Programming Language & Python, Javascript, C\#/C++, Matlab, VB, SQL\\

Software & Solidworks, SW Visualizer, ROS, Altium Designer, Jupyter Lab/Notebook, AWS/Azure, \\& MS-excel, Google Docs and \LaTeX\\  

Interests & Machine Learning, Computer vision, Electronics Systems, Biomedical Devices,\\& Neural Networks, Statistics, Data science, Cloud engineering, Robotics, \\& 3D Modeling and Simulation\\
 
\end{tabular}   

\end{rSection}


\begin{rSection}{publications}

\textbf{Journal Article}\\
\begin{itemize}
    \item El Arja, S.; Jayarathna, T.; Naik, G.; Breen, P.; Gargiulo, G. Characterisation of Morphic Sensors for Body Volume and Shape Applications. Sensors 2020, 20, 90. \href{https://www.mdpi.com/1424-8220/20/1/90}{\textcolor{red}{Paper}}
\end{itemize}

\textbf{Conference Proceeding}\\
\begin{itemize}
    \item S. E. Arja, T. Jayarathna, F. Ulloa, G. Gargiulo and P. Breen, "Characterization of Coated Piezo-resistive Fabric for Respiration Sensing," 2019 International Conference on Electrical Engineering Research & Practice (ICEERP), SYDNEY, Australia, 2019, pp. 1-6, doi: 10.1109/ICEERP49088.2019.8956989. \href{https://ieeexplore.ieee.org/abstract/document/8956989}{\textcolor{red}{Paper}}
\end{itemize}
% \textbf{Unpublished Papers}\\
% \begin{itemize}
%     \item Deep Convolutional Neural Network for Spatial Temporal Human Action Recognition. \href{https://github.com/samiarja/BEngThesis_ResearchPaper/blob/main/BEngThesis_ResearchPaper.pdf}{\textcolor{red}{Paper}}
%     \item Flexible and non-invasive Morphic Sensors for measuring Electrocardiogram (ECG) signals. \href{https://github.com/samiarja/ACURConferencePaper/blob/master/ACURConference2019.pdf}{\textcolor{red}{Paper}}
% \end{itemize}
\end{rSection}
%-------------------------------------------------------------------------------

%	INTERNSHIP/TRAININGS 
%----------------------------------------------------------------------------------------
\begin{rSection}{EXPERIENCE}

\begin{rSubsection}{Software Engineer Intern} {May 2019 - Present}{\faInstitution Nautitech Mining Systems Pty Ltd}{}
\item Build linux based web and desktop application development with C\# for thermal Cameras.
\item Large database design and development using MS SQL Server.
\item Build video compression applications to convert between different protocols (e.g. GIGe, H.264).
\item Build application using OpenCV for image processing.
\item Develop Javascript application to visualize sensors output on the browser.
\item Write engineering technical documentations using MS-word and latex.

\end{rSubsection}

\begin{rSubsection}{Assistant Technical Support Officer} {February 2019 - Present}{\faInstitution Western Sydney University, Engineering \& Industrial Design Cluster}{}
\item Assist technical officers across the electrical, electronics, Mechatronics \& Mechanical team.
\item Assist in lab preparation for classes, basic maintenance duties, moving and managing equipment's.
\item Supporting students in running undergrad lab equipment's.
\end{rSubsection}

\begin{rSubsection}{Undergraduate Researcher} {November 2018 - Present}{\faInstitution The MARCS Institute Brain, Behavior and Development}{}
\item Project: Characterisation of morphic sensors for body volume applications.
\item Supervisors: Dr. Gaetano Garguilo and Dr. Paul Breen.
\item Characterizations of electro-resistive fabrics band to be used in measuring human blood and respiration rate.
\item Implement an accurate data acquisition system on the CC2460R2 module which helps in recognizing patterns for each band during testing. 
\item Using C and Matlab to support the data acquistion, Data analysis and Data visualization.
\item 3D modelling an optimized version of the expansion/contraction machine.
\item Paper Title: "Characterization of Morphic Sensors for body volume and shape applications".
\end{rSubsection}

\begin{rSubsection}{Research Assistant} {May 2018 - May 2019}{\faInstitution CSIRO's Data61}{}
\item Conduct an online research analysis to understand more about business.
\item Review, edit and enter data about Australian businesses providing Environmental Goods and Services.    
\item Classify these Data into specific categories, for further analysis.
\item Use Pandas and Numpy to merge old and new data effectively without losing any entry and perform Data cleaning and Data Munging.
\item Update data already collected against primary sources, as well as identifying and recording data about other relevant organizations.
\end{rSubsection}

\begin{rSubsection}{Robotics Class Instructor} {May 2018 - Present}{\faInstitution RobotZilla - Baulkhaum Hill Public School}{}
\item  Teach kids the principle of coding on Scratch Junior - MIT App Inventor, as well as web development on Python - HTML - CSS - JavaScript.
\item Help Kids to build and program their own Lego Robot.
\item Design robotics kit to support the annual curriculum, such as Robotics arm, quadruped robot and Robotics arm.
\item Platforms used are Scratch, MIT App Inventor, Arduino and Lego Mindstorm.
\end{rSubsection}

\begin{rSubsection}{Robotics Engineer Intern} {October 2017 - January 2018}{\faInstitution Lab38 - Western Sydney University}{}
\item Design a CNC drawing machine using mechanical linear rail to support a frictionless movement.
\item Program the electrical system of the machine to convert any input (image, text) to G-code.
\item Laser cut all parts and components using Corel software, and Solidworks as main program for 3D modelling.
\end{rSubsection}

\begin{rSubsection}{Electronics Engineer Intern} {December 2017 - March 2018}{\faInstitution Mostyn Enterprises}{}
\item  Design Circuit boards on Altium Designer.
\item Generate BOM for each board.
\item Design RF(Radio Frequency) Loop Antenna \& bias tee.
\end{rSubsection}

\end{rSection}  
%-------------------------------------------------------------------------------

%	PROJECTS

\begin{rSection}{PROJECTS}

\begin{rSubsection}{Deep Convolutional Neural Network for Spatial and Temporal 
\\Human Action Recognition} {August 2018 - June 2019}{Thesis - Honour project (High Distinction)}{}

\item Develop a software to extract spatial and temporal features from trimmed video on KTH datasets.
\item Build 2 neural network architecture and compare their performance against the state of the art benchmark results.
\end{rSubsection}  

%------------------------------------------------

\begin{rSubsection}{Fully Autonomous Mobile Robotics system based on Neural Network}{August 2018 - June 2019}{Major final year Capstone project (Distinction)}{} 
\item Model a mobile robot platform that run on raspberry pi and arduino uno and hokuyo Lidar.
\item Develop a software based on convolutional neural network architecture that detect and recognize traffic light and objects in real time.   

\end{rSubsection}

%-------------------------------------------------


\begin{rSubsection}{RobotCup SSL(Small size League)}{November 2017}{Western Sydney University Unlimited Robotics}

\item Design and build a robotics soccer team consists of 11 mobile robots.
\item Use industrial camera to detect each robot pattern and teach the robot to play as team.    

\end{rSubsection}

%-------------------------------------------------- 

\begin{rSubsection}{NI (National Instruments) Autonomous Robotics Competition}{December 2017 – September 2018}{Western Sydney University Unlimited Robotics}{}    

\item Integrate sensors and actuators into a smart mechatronic system. 
\item Develop an algorithm to perform image processing and motion planning
\item Develop navigation and mapping algorithm navigation through tracking features or landmarks within the track.

\end{rSubsection} 

\end{rSection} 

%---------------------------------------------------------------------------

 
 %-----------------------------------------------------------------------------
 % POSITION OF RESPONSIBILITY
 %-----------------------------------------------------------------------------
  
\begin{rSection}{POSITION OF RESPONSIBILITY}

\begin{rSubsection}{Baxter Robot Operator}{February 2018 - Present}{The LaunchPad}{}              
\item Operate and present Baxter Robot in various event: CatalystWest, Semi-permenant, CeBIT.
\end{rSubsection}  

%------------------------------------------------

\begin{rSubsection}{Engineer Team Member} {August 2017 - Present}{Robotics Club - Western Sydney University}{} 
\item Lead in Mechanical work of Robotics club, working mostly with CAD and Hardware systems.
\item Active member working to develop various robots of different concepts and configurations.    
\end{rSubsection}

\end{rSection}
  

%--------------------------------------------------------------------------------------
% Extra-Cirrucular
%--------------------------------------------------------------------------------------

\begin{rSection}{Achievements} \itemsep -2pt   

\begin{itemize}
 
\item \faTrophy \hspace{0.1cm} Top 5 at \textbf{NIARC 2018} under the supervision of Professor Gu Fang. \hfill March 2018 
\item \faTrophy \hspace{0.1cm} Top 25\% at \textbf{Santander Value Prediction Challenge}, a Machine
\\Learning competition hosted by \textbf{Kaggle}.   \hfill May 2018  
\item Leading workshop on \textbf{Introduction to Robotics and Lego Mindstorm}.
\\at Western Sydney University \hfill October 2018 
  
\end{itemize}  


\end{rSection} 

%---------------------------------------------------------------------------------
\end{document}
